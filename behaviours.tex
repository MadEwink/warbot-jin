\documentclass{article}

\usepackage[T1]{fontenc}
\usepackage[utf8]{inputenc}
\usepackage{hyperref}
\usepackage[scale=0.7]{geometry}
\pagenumbering{gobble}

\begin{document}

\section{Stratégie}

\paragraph{}

Notre stratégie générale est centrée autour d'une ruée sur les bases adverses par des Explorers.
Il faut donc trouver les bases adverses le plus vite possible.
Cependant, nous avons aussi besoin d'une agriculture, de manière à continuer de créer des Explorers si la vague initiale est insuffisante.
Il y a donc deux parties à notre stratégie : la gestion des explorers, et celle des harvesters.

\paragraph{}
L'une des parties consiste à créer une ferme autour des bases.
\begin{itemize}
	\item Les Explorers marchent aléatoirement à la recherche de burgers
        \item Les Harvesters suivent l'explorer le plus proche, ou marchent aléatoirement
        \item Quand les Harvesters trouvent de la nourriture, ils la collectent et rentrent à la base, en prévenant les explorers et harvesters en chemin
        \item Quand un Explorer trouve de la nourriture, il rentre à la base, et prévient les explorers et harvesters en chemin
        \item Les Harvesters vont à la source de nourriture s'ils l'a connaissent, et rentrent à la base au bout d'un certain temps s'ils ont errés pendant trop longtemps
        \item Quand les Harvesters arrivent à la base, ils transfèrent une partie de leurs ressources, et plantent le reste
\end{itemize}

\paragraph{}
La partie offensive de notre stratégie consiste à trouver la base ennemie et se ruer dessus avec des explorer.
\begin{itemize}
	\item Des explorers essaient de trouver une base ennemie. Une fois que ceci est fait, une partie rentre à la base, et l'autre partie attaque. La répartition est déterminée aléatoirement.
        \item Les explorers rentrant informent les autres explorers croisés en chemin, qui attaquent alors la base ennemie.
        \item Les explorers qui attaquent foncent tout droit, sans se préoccuper d'éventuels obstacles en chemin.
            La décision de ne pas éviter les obstacles vient du fait que c'est un comportement plus amusant.
            De plus, les attaques suicidaires sont très puissantes, comparées aux missiles, nous n'allons donc pas essayer de pousser notre avantage trop loin.
\end{itemize}

\subsection{Gestion de la mémoire}

Les robots sont gérés comme des machines à état.
La mémoire est gérée différemment selon le type du robot, mais l'emplacement 0 est le même pour les deux, et sert à stocker l'état du robot.

\begin{table}[!htb]
	Mémoire des robots\\
	\begin{tabular}{|c|l|}
		\hline
		Emplacement mémoire & Description\\
		\hline
		Mem 0 a & État actuel du robot\\
		Mem 0 b & Temps depuis le dernier changement d'états\\
		\hline
	\end{tabular}
\end{table}

\section{Gestion des explorers}

\subsection{Gestion de la mémoire}

\begin{table}[!htb]
	Mémoire des explorers\\
	\begin{tabular}{|c|l|}
		\hline
		Emplacement mémoire & Description\\
		\hline
		Mem 1 a & Coordonnée en x pour la nourriture ou une base ennemie x\\
		Mem 1 b & Coordonnée en y pour la nourriture ou une base ennemie\\
		\hline
	\end{tabular}
\end{table}

\subsection{Gestion des états}

\begin{table}[!htb]
	Liste des états\\
	\begin{tabular}{|c|l|}
		\hline
		État & Description \\
		\hline
		0 & Errance : va dans des directions aléatoires, en évitant les obstacles\\
		\hline
		1 & Retour à la base, en transmettant des informations sur la nourriture\\
		\hline
		2 & Recherche de la base ennemie\\
		\hline
		3 & Retour à la base, en transmettant la position de la base ennemie\\
		\hline
		4 & Mode d'attaque, fonce sur la base ennemie tout droit\\
		\hline
	\end{tabular}
\end{table}

\begin{table}[!htb]
	Transitions d'états\\
	\begin{tabular}{|c|c|l|}
		\hline
		De & À & Condition\\
		\hline
		0 & 1 & Ressource à portée\\
		\hline
		1 & 2 & Arrivée à la base\\
		\hline
		2 & 3 & Découverte de la base ennemie, avec une probabilité\\
		\hline
		2 & 4 & Découverte de la base ennemie, avec une probabilité\\
		\hline
	\end{tabular}
\end{table}

\section{Comportement des harvesters}


\subsection{Gestion de la mémoire}

La principale information que doivent stocker les Harvester est la position de source de nourritures.
\newline

\begin{table}[!htb]
	Mémoire des harvesters\\
	\begin{tabular}{|c|l|}
		\hline
		Emplacement mémoire & Description\\
		\hline
		Mem 1 a & Nourriture x\\
		Mem 1 b & Nourriture y\\
		\hline
                Mem 2 & Est-ce que de la nourriture a été trouvée à l'itération précédente\\
                \hline
	\end{tabular}
\end{table}

\subsection{Gestion des états}

Nous avons besoin de quatre états : un pour errer aléatoirement, un autre pour récupérer de la nourriture précise, un pour la ramener à la base, et un dernier pour gérer les fermes.

\begin{table}[!htb]
	Liste des états\\
	\begin{tabular}{|c|l|}
		\hline
		État & Description \\
		\hline
		0 & Errance : va dans des directions aléatoires, évite les obstacles\\
		\hline
		1 & Collecte : Se rend vers une nourriture détectée ou indiquée par un autre robot\\
		\hline
		2 & Rapporte : rapporte sa cargaison de resources à la base\\
		\hline
		3 & Agriculture : Plate des graines et récolte les burgers auour de la base\\
		\hline
	\end{tabular}
\end{table}

Au départ, tous les Harvesters partent explorer.
Cependant, quand ils rentrent à la base déposer des ressources, il y a une probabilité, gérée par un paramètre, que le harvester se mette à gérer les fermes autour de la base.
Après avoir récolté un burger, ils continuent de collecter les autres à sa portée tant qu'il y en a.
\newline

\begin{table}[!htb]
	Transitions d'état pour les Harvesters\\
	\begin{tabular}{|c|c|l|}
		\hline
		De & À & Conditions\\
		\hline
		0 & 1 & Ressources détectées ou emplacement connu grâce à un autre robot\\
		\hline
		1 & 2 & Possède des ressources, et n'en a pas trouvé plus à portée\\
		\hline
		2 & 3 & Est proche de la base\\
		\hline
	\end{tabular}
\end{table}

\section{Comportement des bases}

\subsection{Gestion de la mémoire}

Les bases doivent retenir le nombre d'unités encore à produire.
\begin{table}[!htb]
	Mémoires des bases\\
	\begin{tabular}{|c|l|}
		\hline
		Emplacement mémoire & Description\\
		\hline
		Mem 1 a & Nombre de harvesters encore à produire\\
		Mem 1 b & Nombre d'explorers encore à produire\\
		\hline
	\end{tabular}
\end{table}

\section{RocketLaunchers}

Notre stratégie ne nécessite pas de RocketLaunchers.
Nous n'avons donc pas encore implémenté ces robots.
Cependant, nous avons envisagé les utiliser pour protéger notre base, et donc les faire stationner ou patrouiller à proximité.

\end{document}
