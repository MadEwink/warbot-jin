\documentclass{article}

\usepackage[T1]{fontenc}
\usepackage[utf8]{inputenc}
\usepackage{hyperref}
\usepackage[scale=0.7]{geometry}
\pagenumbering{gobble}

\begin{document}

\section{Stratégie}

\paragraph{}

Notre stratégie générale est centrée autour d'une ruée sur les bases adverses par des Explorers.
Il faut donc trouver les bases adverses le plus vite possible.
Cependant, nous avons aussi besoin d'une agriculture, de manière à continuer de créer des Explorers si la vague initiale est insuffisante.
Il y a donc deux parties à notre stratégie : la gestion des explorers, et celle des harvesters.

\paragraph{}
L'une des parties consiste à créer une ferme autour des bases.
\begin{itemize}
	\item Les Explorers marchent aléatoirement à la recherche de burgers
        \item Les Harvesters suivent l'explorer le plus proche, ou marchent aléatoirement
        \item Quand les Harvesters trouvent de la nourriture, ils la collectent et rentrent à la base, en prévenant les explorers et harvesters en chemin
        \item Quand un Explorer trouve de la nourriture, il rentre à la base, et prévient les explorers et harvesters en chemin
        \item Les Harvesters vont à la source de nourriture s'ils l'a connaissent, et rentrent à la base au bout d'un certain temps s'ils ont errés pendant trop longtemps
        \item Quand les Harvesters arrivent à la base, ils transfèrent une partie de leurs ressources, et plantent le reste
\end{itemize}

\paragraph{}
La partie offensive de notre stratégie consiste à trouver la base ennemie et se ruer dessus avec des explorer.
\begin{itemize}
	\item Des explorers essaient de trouver une base ennemie. Une fois que ceci est fait, une partie rentre à la base, et l'autre partie attaque. La répartition est déterminée aléatoirement.
        \item Les explorers rentrant informent les autres explorers croisés en chemin, qui attaquent alors la base ennemie.
        \item Les explorers qui attaquent foncent tout droit, sans se préoccuper d'éventuels obstacles en chemin.
\end{itemize}

\subsection{Gestion de la mémoire}

Les robots sont gérés comme des machines à état.
La mémoire est gérée différemment selon le type du robot, mais l'emplacement 0 est le même pour les deux, et sert à stocker l'état du robot.

\begin{table}[ht]
	Memory assignments for all robots\\
	\begin{tabular}{|c|l|}
		\hline
		Memory & Description\\
		\hline
		Mem 0 a & Current State\\
		Mem 0 b & Time since last state change\\
		\hline
	\end{tabular}
\end{table}

\section{Gestion des explorers}

\subsection{Gestion de la mémoire}

\begin{table}[ht]
	Memory assignments for explorers\\
	\begin{tabular}{|c|l|}
		\hline
		Memory & Description\\
		\hline
		Mem 1 a & Food x or Enemy base x\\
		Mem 1 b & Food y or Enemy base y\\
		\hline
	\end{tabular}
\end{table}

\subsection{Gestion des états}

\begin{table}[ht]
	State array\\
	\begin{tabular}{|c|l|}
		\hline
		State & Description \\
		\hline
		0 & Wandering : goes in random directions, while trying to dodge objects\\
		\hline
		1 & Goes back to base to transmit food info\\
		\hline
		2 & Searching enemy base\\
		\hline
		3 & Goes back to base to transmit enemy base location\\
		\hline
		4 & Suicide mode\\
		\hline
	\end{tabular}
\end{table}

\begin{table}[ht]
	State transitions\\
	\begin{tabular}{|c|c|l|}
		\hline
		From & To & Conditions\\
		\hline
		0 & 1 & Resource in range\\
		\hline
		1 & 2 & Arrived to base\\
		\hline
		2 & 3 & Find enemy base and random\\
		\hline
		2 & 4 & Find enemy base and random\\
		\hline
	\end{tabular}
\end{table}

\section{Harvesters behaviour}


\subsection{Gestion de la mémoire}

La principale information que doivent stocker les Harvester est la position de source de nourritures.
\newline

\begin{table}[ht]
	Memory assignments for harvesters\\
	\begin{tabular}{|c|l|}
		\hline
		Memory & Description\\
		\hline
		Mem 1 a & Food x\\
		Mem 1 b & Food y\\
		\hline
                Mem 2 & Has food been found the previous turn\\
                \hline
	\end{tabular}
\end{table}

\subsection{Gestion des états}

Nous avons besoin de quatre états : un pour errer aléatoirement, un autre pour récupérer de la nourriture précise, un pour la ramener à la base, et un dernier pour gérer les fermes.

\begin{table}[ht]
	State array\\
	\begin{tabular}{|c|l|}
		\hline
		State & Description \\
		\hline
		0 & Wandering : goes in random directions, while trying to dodge objects\\
		\hline
		1 & Collecting : goes to a resource location known or indicated by an Explorer\\
		\hline
		2 & Retrieving : brings a resource bask to the base\\
		\hline
		3 & Farming : Plant seeds and harvest seeds around base\\
		\hline
	\end{tabular}
\end{table}

Au départ, tous les Harvesters partent explorer.
Cependant, quand ils rentrent à la base déposer des ressources, il y a une probabilité, gérée par un paramètre, que le harvester se mette à gérer les fermes autour de la base.
\newline

\begin{table}[ht]
	Transitions d'état pour les Harvesters\\
	\begin{tabular}{|c|c|l|}
		\hline
		From & To & Conditions\\
		\hline
		0 & 1 & Resource or Explorer knowing resource in range\\
		\hline
		1 & 2 & Carries resource\\
		\hline
		2 & 3 & Near base\\
		\hline
	\end{tabular}
\end{table}

\section{Bases behaviour}

\subsection{Gestion de la mémoire}

Les bases doivent retenir le nombre d'unités encore à produire.
\begin{table}[ht]
	Memory assignments for bases\\
	\begin{tabular}{|c|l|}
		\hline
		Memory & Description\\
		\hline
		Mem 1 a & Number of harvesters still to be produce\\
		Mem 1 b & Number of explorers still to be produced\\
		\hline
	\end{tabular}
\end{table}

\section{RocketLaunchers}

Notre stratégie ne nécessite pas de RocketLaunchers.
Nous n'avons donc pas encore implémenté ces robots.
Cependant, nous avons envisagé les utiliser pour protéger notre base, et donc les faire stationner ou patrouiller à proximité.

\end{document}
